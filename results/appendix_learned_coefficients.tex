\section{Learned Coefficient Values}
\label{app:learned_coefficients}

Table~\ref{tab:learned_coefficients} reports the average learned coefficients for each mergeability metric across all merging methods. These coefficients are obtained by averaging across all 20 folds of the leave-one-task-out cross-validation procedure. The coefficients operate on min-max normalized metrics (scaled to $[0, 1]$), so their magnitudes reflect both the importance of each metric and its original scale.

\paragraph{Interpretation.}
A positive coefficient indicates that higher values of the corresponding metric predict better post-merge performance, while a negative coefficient indicates the opposite relationship. The magnitude of a coefficient reflects how strongly that metric influences the prediction, though direct comparison across metrics requires accounting for their different variances.

\paragraph{Key Observations.}
Several patterns emerge from the coefficient values:

\begin{itemize}
    \item \textbf{Gradient-based metrics} receive consistently large-magnitude coefficients across methods, with input gradient L2 distance showing strong negative coefficients for all methods (ranging from $-25.0$ to $-42.3$). This suggests that large gradient differences between task vectors are detrimental to merging success.

    \item \textbf{Distance metrics} (L2 distances) generally receive negative coefficients, indicating that greater dissimilarity between models predicts worse merging outcomes. Conversely, similarity metrics (cosine similarity, overlap measures) tend to receive positive coefficients.

    \item \textbf{Method-specific patterns}: Weight Averaging shows distinctively large coefficients for right subspace overlap ($41.5$) and input gradient cosine similarity ($45.0$), suggesting these metrics are particularly informative for this method. Task Arithmetic and Isotropic show similar coefficient patterns, while TSV exhibits generally smaller coefficient magnitudes.

    \item \textbf{Sign consistency}: Some metrics show consistent signs across all methods (e.g., input gradient L2 distance is always negative), while others flip signs depending on the merging method, highlighting the importance of method-specific prediction models.
\end{itemize}

\begin{table}[htbp]
\centering
\caption{Average learned coefficients for each mergeability metric across merging methods, obtained via leave-one-task-out cross-validation. Positive coefficients indicate that higher metric values predict better post-merge performance, while negative coefficients indicate the opposite. Coefficients are on normalized metrics (min-max scaled to $[0, 1]$).}
\label{tab:learned_coefficients}
\resizebox{\textwidth}{!}{%
\begin{tabular}{clrrrr}
\toprule
 & \textbf{Metric} & \textbf{Task Arithmetic} & \textbf{Weight Averaging} & \textbf{Isotropic} & \textbf{TSV} \\
\midrule
 & TV Cosine Sim & 5.31 & 25.77 & 11.68 & -6.19 \\
 & TV L2 Dist & -20.00 & 7.54 & -19.20 & -1.38 \\
\multirow{5}{*}{\rotatebox{90}{\textbf{Task Vector Geometry}}} & TV Dot Prod & 0.19 & 32.70 & 8.68 & -0.02 \\
 & Weight Angle & 4.81 & 9.04 & 3.89 & 2.26 \\
 & TV Mag Ratio & -3.46 & 3.61 & -4.70 & -4.59 \\
\midrule
 & Eff Rank & 0.77 & -15.86 & -18.38 & -4.00 \\
 & Eff Rank Score & 0.65 & -22.59 & 7.26 & -10.86 \\
 & Stable Rank & -11.98 & -9.09 & -13.51 & -8.80 \\
\multirow{7}{*}{\rotatebox{90}{\textbf{Effective Rank}}} & Spectral Gap & 2.61 & 7.01 & -13.05 & -5.88 \\
 & SV Ratio & -3.74 & -26.14 & -10.78 & 5.65 \\
 & Layer Eff Rank & -9.34 & -11.63 & 14.81 & 0.35 \\
 & Layer Eff Rank Score & -9.40 & 7.37 & -3.62 & 10.91 \\
\midrule
 & SV Overlap & -21.14 & 11.63 & -27.66 & 9.72 \\
 & Left Sub Top-$k$ & 6.93 & -2.58 & 22.26 & 11.17 \\
 & Right Sub Top-$k$ & 15.25 & 24.88 & 16.71 & 10.90 \\
\multirow{6}{*}{\rotatebox{90}{\textbf{Subspace Overlap}}} & Right Sub Bot-$k$ & 2.76 & -21.46 & 13.11 & -0.86 \\
 & Interact Top-$k$ & 21.56 & 3.05 & 16.15 & 13.93 \\
 & Interact Bot-$k$ & 26.45 & -7.69 & 31.37 & -9.84 \\
\midrule
 & Act L2 Dist & 0.79 & -30.62 & -10.56 & -1.44 \\
 & Act Cosine Sim & 10.00 & -1.37 & 8.94 & -0.10 \\
\multirow{4}{*}{\rotatebox{90}{\textbf{Activation-Based}}} & Act Mag Ratio & -13.82 & -44.81 & -2.69 & -4.79 \\
 & Act Dot Prod & 14.59 & 21.18 & -6.50 & 4.47 \\
\midrule
 & Enc Grad Cos & 16.36 & -5.07 & 22.76 & 11.90 \\
 & Enc Grad L2 & -26.21 & -13.82 & -25.24 & -5.01 \\
 & Enc Grad Dot & 20.15 & 21.24 & 14.64 & 3.34 \\
\multirow{6}{*}{\rotatebox{90}{\textbf{Gradient-Based}}} & Input Grad Cos & 15.08 & 44.99 & 15.52 & 6.01 \\
 & Input Grad L2 & -31.53 & -38.55 & -42.31 & -25.02 \\
 & Input Grad Dot & -28.27 & -9.29 & -24.26 & -1.77 \\
\bottomrule
\end{tabular}%
}
\end{table}